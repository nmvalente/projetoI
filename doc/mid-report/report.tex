\documentclass[a4paper]{article}
\usepackage[english]{babel}
\usepackage[utf8x]{inputenc}
\usepackage{float}
\usepackage{graphicx}
\usepackage{caption}
\usepackage{pmboxdraw}
\usepackage{color}
\usepackage{tocloft}
\usepackage[color]{vdmlisting}
\usepackage{longtable}
\usepackage[hidelinks]{hyperref} 
\usepackage{geometry}
\geometry{a4paper, total={160mm,225mm}, left=25mm, top=35mm}

\definecolor{darkgray}{rgb}{0.41, 0.41, 0.41}
\definecolor{green}{rgb}{0.0, 0.5, 0.0}

\usepackage{listingsutf8}
\lstdefinestyle{JavaStyle}{
	language=Java, 	
	numbers=left,
	stepnumber=5,
	firstnumber=1,
	numberfirstline=true,
    basicstyle=\linespread{0.7}\ttfamily,
    keywordstyle=\color{blue}\ttfamily,
	showstringspaces=false,
    stringstyle=\color{red}\ttfamily,
    commentstyle=\color{green}\ttfamily,
	identifierstyle=\color{darkgray}\ttfamily,
	tabsize=4,
    breaklines=true,
    extendedchars=true,
	inputencoding=utf8x,
    escapeinside={\%*}{*)},
	frame=lines
}

\setlength{\tabcolsep}{6pt}
\setlength\cftaftertoctitleskip{20pt}
\begin{document}
%\setlength{\textwidth}{16cm}
%\setlength{\textheight}{22cm}


\title{\includegraphics[scale=0.15]{feup_logo.png}
\linebreak\linebreak\linebreak\linebreak\linebreak
\Large\textbf{Research applied to the collection of waste in a city }\linebreak  {\large Relatório Intercalar}
\linebreak\linebreak\linebreak\linebreak
\Large{Inteligência Artificial - 3\textsuperscript{rd} degree}
\linebreak
\Large{Mestrado Integrado em Engenharia Informática e Computação}\linebreak\linebreak\linebreak\linebreak
}
\author{\textbf{Turma 3 - Grupo A3\_2}\\
Artur Ferreira - ei12168 - ei12168@fe.up.pt\\
Nuno Valente - up200204376 - up200204376@fe.up.pt\\
\linebreak\linebreak \\
\linebreak\linebreak\linebreak
\linebreak\linebreak\vspace{1cm}}

\maketitle

\thispagestyle{empty}
\newpage
\tableofcontents 
\newpage

\section{Objective}
\label{objective}

This project aims to determine the best route to be performed by a collection of waste trucks in a city, and it has two main objectives: minimizing the distance travelled on the route taken and maximizing the load waste transported.

\section{Description}
\label{description}

Waste collection is a daily task in a city that must be performed as efficiently as possible, either to keep the city clean or to minimize the associated costs. In order to transport waste to the treatment stations, the city services maintain a fleet of specialized lorries which carry out collection routes, that are defined previously and carried out systematically at a given frequency.

It is intended to perform such collection more intelligently. In fact, containers scattered in various parts of the city, where the residents deposit the garbage. These containers may not be full enough to justify emptying them by the collection truck, which would make some trips unnecessary. With the technology of sensor networks developing rapidly, more effective monitoring of the level of waste accumulation in each container is already possible.

We have considered the existence of 4 types of waste: paper, plastic, glass and ordinary trash. Each truck carries only one type of waste, because we must think in recycling.

In this work, we intend to develop an application that determines the collection routes to be made by trucks, considering only the containers with sufficient residue that justifies their collection. This application should be able to suggest the best route (minimization of the distance traveled and maximization of the transported waste) from the central, where the trucks are stationed, to the treatment stations, where all the collected waste is deposited.

As a first step, we have considered that the collection is carried out by a single truck of limited capacity. In a second phase, we'll consider that there are several trucks with limited capacity and when trying to optimize the route, we want to use as few trucks as possible.

\subsection{Specification}
\subsubsection{Important concepts}

In this problem we need to consider a few concepts like truck, container, place of departure, place of arrival and the desired route. More properly:

\begin{itemize}
	\item The specialized truck has a limited capacity and a type of waste;
	\item The place of departure is the central where are the trucks to initialize their route;
	\item The place of arrival is where are the treatment stations that the trucks use to leave the waste collected;
	\item One container is consider as a set of four individual type waste;
	\item The itinerary that we're trying to determine considering the objective. 
\end{itemize}

\subsubsection{Problem description}

In a summarized way, we need to determine the best itinerary that contemplates the already referred objectives in section \ref{objective}.

\subsubsection{Problem restrictions}

In order to make the problem more realistic, we try to use some streets of Oporto city, with real latitude and longitude coordinates where we put the containers. In this sense, it is intended that in a more advanced phase of the project, these coordinates are used to better represent the selected itinerary and the distance covered. 

\subsubsection{Problem representation}

To represent the map of this problem, we considered a non-directed graph with a list of adjacent edges on each node and it is used to represent, in general, the garbage container map. A node represents a point of passage: central, container or treatment station. The edges of the graph store the distance between the node that has that edge and the target node.

\subsubsection{List of requirements}\label{subsecsec:requirements}

\begin{table}[H]
	\centering
	\label{tab:list-of-requirements}
	\begin{tabular}{|l|l|p{9cm}|}
		\hline
		\textbf{Id} & \textbf{Priority} & \textbf{Description} \\ \hline
		R1 & Mandatory        & The user can chose the number of each type of truck available on the central  \\ \hline
		R2 & Mandatory         & The user can enter the truck capacity \\ \hline
		R3 & Mandatory          & The user can select the number of stations to leave the garbage \\ \hline
		R4 & Mandatory         & The user will see the result of the implemented search algorithm in console \\ \hline
		R5 & Mandatory         & The application must provide the result with the data that the user chose to test \\ \hline
		R6 & Opcional         & The user will see the result of the implemented search algorithm in a graphical friendly user interface.\\ \hline
		R7 & Opcional         & Nodes and edges are loaded from a csv file to facilitate the edition of data \\ \hline
		R8 & Opcional         & The user might chose other algorithms to find the best itinerary \\ \hline
	\end{tabular}
\end{table}

\subsubsection{Solution}\label{solution}

In order to finding the best solution to the problem, we intend to apply the algorithm $A^*$ to a graph that represents a map with some of the possible routes from the Central to the Treatment Stations. This algorithm figures the least cost path to the node which makes it a best first search algorithm. Uses the following formula:

\begin{equation}\label{equation}
f(x) = g(x) + h(x)
\end{equation}

where $g(x)$ is the total distance in kilometers ($km$) from the initial position to the current position, $h(x)$ is the heuristic function that is used to approximate distance from the current location to the goal state, minimize the number of active trucks and maximize the waste loaded. This function is distinct because it is a mere estimation rather than an exact value. The more accurate the heuristic the better and faster the goal state is reach and with much more accuracy.

The equation refered in (\ref{equation}) represents the current approximation of the shortest path to the goal. 

To ensure the admissibility of the $h(x)$ heuristic function, it is thought to calculate the straight-line distance between the geographic coordinates between the node in question and the last node - the station, only when all containers have been reached, of course, if they have enough garbage to collect.

\subsection{Work already done}

We believed that the next exposed method to approach the problem, facilitate our work. First, we make a effort to understand the problem. After and considering what we learned on lectures classes, we analyse and we made a choice of one of the problem solving methodologies - using algorithms for informed solutions research. 
Posteriorly we felt the need to have some data and store a little bank of data with information of nodes and edges in one file \emph{.csv}.
Further, we made prints to have a simple debug implemented. 
In our last step until the delivery of this report we are trying to implement the algorithm already referred in the last subsection \ref{solution}.

Nodes and edges were generated from coordinates obtained by OpenStreetMap, an open source tool, where we export a rectangle zone of streets from city of Oporto. 

In our approach we are, gradually, increasing the level of difficult to reach the objective of the project. Until know, we can chose the desired number of each type of trucks, the truck capacity and the number of stations, all in a graphical interface. For applying the algorithm, in this phase, we only have one truck of paper waste and one treatment station, although the graphical interface is already prepared to edit with other data. 

We also have one basic other graphical interface prepared to show the results of the algorithm.


The best solution to the problem was calculated by counting the distance as a calculation factor for both the cost function up to the $g(x)$ node and the heuristic function $h(x)$. 

Finally we've done a basic graphical representation of the initial map, the solution generated by the application and implemented classes to test in JUnit.

It was used a website that has an online tool that allow us to generate real coordinates.\ref{itm:openStreet}

\subsection{Expected results and future evaluation of the project}

In the future we need to validate the results. Why? Because of two reasons: software always has defects and we want to catch them before we deliver the project. It is important to us that the application is consistent and we have some degree of safe with what we developed; by other hand, we want to test the solution, more properly, the implemented algorithm. 

Other thing we want to permit the user to do is to let him visualize the solution in a friendly way, with a graph viewer tool.

\section{Conclusions} 

The application developed until now meets some items of the list of requirements \ref{subsecsec:requirements}, but in a small scale, because the number of data used. We want to increment the number of nodes and edges, maybe try and test with other algorithm to compare the solutions. 

Last we can say that we have one doubt - what graph viwer tool we'll use and if it is a good tool. 

\newpage

\section{Resources}

\subsection{References}

\begin{enumerate}
	
	\item Slides from lectures classes
	\bibitem{lamport93}
	Stuart Russell, Peter Norvig
	\emph{Artifical Intelligence A Modern Approach},
	Pearson Education
	3rd edition,
	2010.
	\item \url{http://www.gpsvisualizer.com/tutorials/waypoints.html} 
	\item \url{https://reference.wolfram.com/language/ref/ManhattanDistance.html}
	\item \url{http://junit.org/junit4/}
	

\end{enumerate}

\subsection{Used software} \label{openStreet}
\begin{enumerate}
	
\item \url{http://www.openstreetmap.org/}\label{itm:openStreet}
\item \url{http://jgrapht.org/}

\end{enumerate}

\newpage
\appendix

%\section{Annex}\label{}
	
\end{document}
