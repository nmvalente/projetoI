\documentclass[a4paper]{article}
\usepackage[english]{babel}
\usepackage[utf8x]{inputenc}
\usepackage{float}
\usepackage{graphicx}
\usepackage{caption}
\usepackage{pmboxdraw}
\usepackage{color}
\usepackage{tocloft}
\usepackage[color]{vdmlisting}
\usepackage{longtable}
\usepackage[hidelinks]{hyperref} 
\usepackage{geometry}
\geometry{a4paper, total={160mm,225mm}, left=25mm, top=35mm}

\definecolor{darkgray}{rgb}{0.41, 0.41, 0.41}
\definecolor{green}{rgb}{0.0, 0.5, 0.0}

\usepackage{listingsutf8}
\lstdefinestyle{JavaStyle}{
	language=Java, 	
	numbers=left,
	stepnumber=5,
	firstnumber=1,
	numberfirstline=true,
    basicstyle=\linespread{0.7}\ttfamily,
    keywordstyle=\color{blue}\ttfamily,
	showstringspaces=false,
    stringstyle=\color{red}\ttfamily,
    commentstyle=\color{green}\ttfamily,
	identifierstyle=\color{darkgray}\ttfamily,
	tabsize=4,
    breaklines=true,
    extendedchars=true,
	inputencoding=utf8x,
    escapeinside={\%*}{*)},
	frame=lines
}

\setlength{\tabcolsep}{6pt}
\setlength\cftaftertoctitleskip{20pt}
\begin{document}
%\setlength{\textwidth}{16cm}
%\setlength{\textheight}{22cm}


\title{\includegraphics[scale=0.15]{feup_logo.png}
\linebreak\linebreak\linebreak\linebreak\linebreak
\Large\textbf{Research applied to the collection of waste in a city }\linebreak  {\large Relatório Intercalar}
\linebreak\linebreak\linebreak\linebreak
\Large{Inteligência Artificial - 3\textsuperscript{rd} degree}
\linebreak
\Large{Mestrado Integrado em Engenharia Informática e Computação}\linebreak\linebreak\linebreak\linebreak
}
\author{\textbf{Grupo ?? Turma ???}\\
Artur Ferreira - ei12168 - ei12168@fe.up.pt\\
Nuno Valente - up200204376 - up200204376@fe.up.pt\\
\linebreak\linebreak \\
\linebreak\linebreak\linebreak
\linebreak\linebreak\vspace{1cm}}

\maketitle

\thispagestyle{empty}
\newpage
\tableofcontents 
\newpage

\section{Objective}
\label{objective}

This project aims to determine the best route to be performed by a collection of waste trucks in a city, and it has two main objectives: minimizing the distance travelled on the route taken and maximizing the load waste transported.

\section{Description}
\label{description}

Waste collection is a daily task in a city that must be performed as efficiently as possible, either to keep the city clean or to minimize the associated costs. In order to transport waste to the treatment stations, the city services maintain a fleet of specialized lorries which carry out collection routes, that are defined previously and carried out systematically at a given frequency.

It is intended to perform such collection more intelligently. In fact, containers scattered in various parts of the city, where the residents deposit the garbage. These containers may not be full enough to justify emptying them by the collection truck, which would make some trips unnecessary. With the technology of sensor networks developing rapidly, more effective monitoring of the level of waste accumulation in each container is already possible.

We have considered the existence of 4 types of waste: paper, plastic, glass and ordinary trash. Each truck carries only one type of waste, because we must think in recycling.

In this work, we intend to develop an application that determines the collection routes to be made by trucks, considering only the containers with sufficient residue that justifies their collection. This application should be able to suggest the best route (minimization of the distance traveled and maximization of the transported waste) from the central, where the trucks are stationed, to the treatment stations, where all the collected waste is deposited.

As a first step, we have considered that the collection is carried out by a single truck of limited capacity. In a second phase, we'll consider that there are several trucks with limited capacity and when trying to optimize the route, we want to use as few trucks as possible.

\subsection{Specification}
\subsubsection{Important concepts}

In this problem we need to consider a few concepts like truck, container, place of departure, place of arrival and the desired route. More properly:

\begin{itemize}
	\item The specialized truck has a limited capacity and a type of waste;
	\item The place of departure is the centrar where are the trucks to initialize their route;
	\item The place of arrival is where are the treatment stations that the trucks use to leave the waste collected;
	\item One container is consider as a set of four individual type waste;
	\item And the itinerary that we're trying to determine considering the objective. 
\end{itemize}

\subsubsection{Problem description}

In a summarized way we need to determine the best itinerary that contemplates the already referred objectives in \ref{objective}.

\subsubsection{Problem restrictions}

In order to make the problem more realistic, we try to use some streets of the city of Oporto, with coordinates of real latitude and longitude. In this sense, it is intended that in a more advanced phase of the project, these coordinates are used to better represent the selected itinerary. 

\subsubsection{Problem representation}

To represent the map of this problem, we considered a non-directed graph with a list of adjacent edges on each node and it is used to represent, in general, the garbage container map. A node represents a point of passage: central, container or treatment station. The edges of the graph store the distance between the node that has that edge and the target node.

\subsubsection{Solution}

% todo

Representação do tema como problema de pesquisa: estados, função de transição, heurísticas.
Algoritmos de pesquisa a aplicar (ilustrados para o caso concreto).

\subsection{Work already done}

Utilizou-se página web que possui uma ferramenta online \ref{openStreet}

\subsection{Expected results and future evaluation of the project}

\section{Conclusions} 

The model that we developed covers all the requirements included implicitly on the theme project and the list of requirements described in section \ref{sec:requirements}.

In the end, and after model verifications, we all see the game developed in VDM++ like one of the more consistent and safe projects that we have ever developed during the course.

Even though we all had previous contact with several versions of the game, we are quite pleased with this version of ours. The main things we would like to change would have to be done in Java, because they are related with the Command Line Interface. Indeed, we wanted to make the game accept keystrokes without the need to press Enter. And if that had been achieved, we would have implemented a timer that depending on the level would make the pieces move down in increasing speeds, making the game more difficult. It would also be interesting to show where each piece will land (ghost piece function) and the next tetramino that will appear. However, we feel that these features, lay outside the spectrum of the course.

Overall the project took approximately 80 man-hours to develop. We feel that the work was more or less well distributed between the members, and although each one was mainly focused on a different part of the project, we discussed most things with each other and planned together.

\section{Resources}

\subsection{References}

\begin{enumerate}
	
	\item https://en.wikipedia.org/wiki/Tetris
	
\end{enumerate}

\subsection{Used software} \label{openStreet}
\begin{enumerate}
	
\item http://www.openstreetmap.org/

\end{enumerate}

\newpage
\appendix

\section{Annex}\label{}




	
\end{document}
